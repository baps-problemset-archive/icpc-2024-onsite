\section*{Problem C. Cut the Stick, Share You Must}
Problem Setter: Rumman Mahmud \\
Tester: Shahjalal Shohag, Pritom Kundu, Nafis Sadique \\
Cetegory: Math, Number Theory \\
Total Solved: 15 \\
First to Solve: IUT\_CocolaChampionBiscuit \\
\\
Since $n$ \& $k$ are large ($1 \le k < n \le 10^6$), we need to find an optimal solution.
Here are some hints before we dive into the actual solution.
Let’s define a function $P(n,k)$ that returns the number of ways to split $n$ into $k$ partitions 
where the GCD of the partitions is a prime number.
\begin{itemize}
  \item \bf{Hint 1:} How would you solve it if the GCD was $1$ instead of a prime number?
  \item \bf{Hint 2:} How does changing the GCD to a prime affect $n$?
  \item \bf{Hint 3:} Define $f(n,k)$ to return the number of ways to split $n$ into $k$
    partitions where the GCD of the partitions is $1$. Define another function $g(n,k,p)$ to
    return the number of ways to split $n$ into $k$ partitions where the GCD of the partitions
    is exactly $p$ (with $p$ being prime). Can you draw a relation between $g(n,k,p)$ and $f(\frac{n}{p},k)$?
\end{itemize}

{\bf{Solution}} \\
A valid split will have a prime GCD. Let’s say the GCD of a valid sequence is $p$. 
This means each partition will be a multiple of $p$. If we divide each partition by $p$, 
the new sequence will have GCD $1$. The GCD cannot be greater than $1$, 
as otherwise, the GCD of the original sequence cannot equal $p$. 
Using this observation, $g(n,k,p) = f(\frac{n}{p},k)$. \\
To compute $P(n,k)$, sum over all prime factors of $n$:
$$P(n,k) = \sum_{p \in U(n)} f\left(\frac{n}{p},k\right)$$
Here, $U(n)$ is the set of unique prime factors of $n$. 
The remaining challenge is finding the number of ways to split 
$n$ into $k$ partitions with a GCD of $1$. The Möbius function 
and binomial coefficients can be used for this computation. 
Details are left as an exercise. \\
\\
{\bf{Complexity}}
\begin{itemize}
  \item \bf{Precomputing unique prime factors of all numbers:} $O(N \log N)$
  \item \bf{Precomputing factorial and inverse factorials modulo $M$:} $O(N + \log M)$
  \item \bf{Complexity of finding valid partitions per test case:} 
  For each prime factor $p_i$, finding splits with GCD $1$ has a complexity of 
  $O(2^{\omega(\frac{n}{p_i})})$. Repeating this for all unique prime 
  factors of $n$ gives $O(\sum_{p \in U(n)} 2^{\omega(\frac{n}{p})})$, 
  where $\omega(n)$ is the number of unique prime factors of $n$.
  \item \bf{Overall complexity:} $O(N \log N + T \cdot \sum_{p \in U(n)} 2^{\omega(\frac{n}{p})})$
\end{itemize}
The maximum number of unique primes is $7$. For example, the product of the first 
$7$ primes is $2 \cdot 3 \cdot 5 \cdot 7 \cdot 11 \cdot 13 \cdot 17 = 510510$. 
Adding another prime would exceed $N$. Hence, the maximum value of 
$\sum_{p \in U(n)} 2^{\omega(\frac{n}{p})}$ is $7 \cdot 2^6 = 448$. 
Which means, per query you will need to perform 448 operations in the worst case. 
Which is good enough for the TL. \\
\\
{\bf{Bonus}} \\
We can further reduce the complexity by only considering the 
divisors of n that have at most one square prime factor. 
We need to modify the Mobius function a bit to achieve that. 
To not spoil the fun, I will keep it as an exercise for you as well.
