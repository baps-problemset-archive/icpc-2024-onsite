\section*{Problem H: Hand Cricket}
Problem Setter: Kazi Md Irshad \\
Tester: Pritom Kundu, Jubayer Rahman, Nafis Sadique \\
Category: Data Structure, Math, Probability \\
Total Solved: 0 \\
First to Solve: N/A \\
\\
The strategies are in \href{https://en.wikipedia.org/wiki/Nash_equilibrium}{Nash equilibrium}. Therefore, Alice cannot improve her strategy. She will have the same expected points for any index. Let this expected value be $X$. Let $p_1, p_2, \dots, p_N$ be Bob's strategy. Then for all $i$:
$$ A_i \cdot (1 - p_i) = X $$
$$p_i = 1 - \frac{X}{A_i}$$
$$1 = \sum_{i=1}^N(1-\frac{X}{A_i})$$
As sum of $p_i$ is 1. Adding this equation over all $i$ gives:
$$ X = \frac{N - 1}{\sum_{i=1}^N \frac{1}{A_i}}$$ 
Which means for range $[L,R]$ the answer is $X = \frac{R-L}{\sum_{i=L}^R \frac{1}{A_i}}$
Alice needs to increase $X$. Reducing $\sum_{i=1}^N \frac{1}{A_i}$ will increase $X$. It can be proved that Alice always increments the smallest $A_i$. To implement this efficiently, we will build a persistent segment tree on the array values. The $j^{th}$ root of the segment tree will account for array elements in the range $[1, j]$. The tree maintains the following:
\begin{itemize}
    \item The sum of all $A_i$ values in the range.
    \item The sum of inverses of $A_i$ in the range.
    \item The count of elements available in the range.
\end{itemize}
We will walk the segment tree from the root for the $(L-1)^{th}$ element and the root of the $R^{th}$ element. During this traversal, we will maintain:
\begin{itemize}
    \item How much of $K$ is left.
    \item The value and count of the small numbers that have been incremented.
\end{itemize}
On a segment tree node we can check if the whole subtree can be incremented to its maximum in $O(1)$ and also get the sum of inverses in $O(1)$ if $K$ depletes.

The complexity is $O(QlogN)$. However we have allowed slower solutions like $O(Qlog^2N)$ (using binary search and persistent segment tree), and $O(Qlog^3N)$ (using binary search and merge sort tree).
