\section*{Problem K: Packet Transmission}
Problem Setter: Ashraful Islam \\
Tester: Pritom Kundu, Nafis Sadique \\
Cetegory: Least Common Ancestor \\
Total Solved:  \\
First to Solve: \\
\\
Every query we receive can be classified in the following way.
\begin{itemize}
    \item The query packets (source, destination pair) share a common path in the tree. 
    \begin{itemize}
        \item The packets are going in the same direction. In that case we need to assume one of the
        packets will never wait. So, the other packet will arrive at the first vertex of the common path,
        wait for the first packet to arrive and then go after it. Since an edge can't be used by
        multiple packets at the same time, the other packet will have to wait until the first packet 
        crossed that edge. Keep doing that until they reach the end of common path and then they can 
        go their own ways. If we think carefully, we can see that the other packet will have to wait
        an additional time totalling the maximum edge cost on the common path. However if the first
        packet comes early then the wait time reduces. Do the same the other way around and take
        the minimum of the two.
        \item The packets are going in the different directions. So, we need both packets to start
        their journey. Eventually they might meet on the two sides of an edge where if one starts crossing,
        the other must wait. Make one of them wait and the other one go and calculate the time it would 
        take for both them to reach their destinations. Do it both ways and take the minimum time. However if
        they don't meet at the edge then they don't have to stop and their actual time to reach both 
        their destinations is the answer. 
    \end{itemize}
    \item The query packets use completely different paths. In that case the time it takes for both to reach
    their destination is the answer.
\end{itemize}
We can calculate all of these using Sparse Table to calculate Least Common Ancestors in a tree. Alternatively 
we can use heavy-light decomposition, but that may be very slow. There are lot of corner cases, so careful 
implementation is necessary.
